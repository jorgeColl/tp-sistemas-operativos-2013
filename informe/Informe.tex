%%%%%%%%%%%%%%%%%%%%%%%%%%%%%%%%%%%%%%%%%%%%%%%%%%%%%%%%%%%%%%%%%%%%%%%%%%%%%%%
% Definici�n del tipo de documento.                                           %
% Posibles tipos de papel: a4paper, letterpaper, legalpapper                  %
% Posibles tama�os de letra: 10pt, 11pt, 12pt                                 %
% Posibles clases de documentos: article, report, book, slides                %
%%%%%%%%%%%%%%%%%%%%%%%%%%%%%%%%%%%%%%%%%%%%%%%%%%%%%%%%%%%%%%%%%%%%%%%%%%%%%%%
\documentclass[a4paper,10pt]{article}

%%%%%%%%%%%%%%%%%%%%%%%%%%%%%%%%%%%%%%%%%%%%%%%%%%%%%%%%%%%%%%%%%%%%%%%%%%%%%%%
% Los paquetes permiten ampliar las capacidades de LaTeX.                     %
%%%%%%%%%%%%%%%%%%%%%%%%%%%%%%%%%%%%%%%%%%%%%%%%%%%%%%%%%%%%%%%%%%%%%%%%%%%%%%%

% Paquete para inclusi�n de gr�ficos.
\usepackage{graphicx}
\usepackage{verbatim}

% Paquete notas al pie de p�gina
\usepackage{footnote}

% Paquete que linkea el indice

\usepackage{hyperref}

\hypersetup{
    colorlinks,
    citecolor=black,
    filecolor=black,
    linkcolor=blue,
    urlcolor=black
}

% Agregar indice en el panel lateral.

\usepackage{ifpdf}
%if output to PDF then put the following in PDF header
\ifpdf  
    \pdfcatalog { /PageMode (/UseOutlines)
                  /OpenAction (fitbh)  }
\fi



% Paquete para definir la codificaci�n del conjunto de caracteres usado
% (latin1 es ISO 8859-1).
\usepackage[latin1]{inputenc}

% Paquete para definir el idioma usado.
\usepackage[spanish]{babel}

\usepackage{listingsutf8}
\usepackage{listings}
\lstset{language={}, inputencoding=utf8/latin1, tabsize=4, showstringspaces=false, breaklines=true}

% T�tulo principal del documento.
\title{	{\Huge{	\textbf{Trabajo Pr�ctico: \\
		     Sistma de reserva de entradas sin costo comercial \\ }}}
		    {\hspace{30pt}}}

% Informaci�n sobre los autores.
\author{
            \normalsize{Grupo N� 8 TEMA B} \\
            \normalsize{2do. Cuatrimestre de 2013} \\
            \normalsize{75.08 Sistmas Operativos}                             \\
            \normalsize{Facultad de Ingenier�a, Universidad de Buenos Aires}            \\
       }
\date{}

\begin{document}

% Inserta el t�tulo.
\maketitle

% Quita el n�mero en la primera p�gina.
\thispagestyle{empty}

% Empieza en una pagina nueva.
\newpage

\tableofcontents

\newpage

\section{Hip�tesis y Aclaraciones globales.}


\newpage



\section {Problemas Relevantes}

\newpage

\section{Archivos Readme}

\subsection{Readme\_Instalar}

\lstinputlisting{../Readme_Instalar}

\subsection{Readme\_Iniciar}

\lstinputlisting{../Readme_Iniciar}

\subsection{Readme\_Recibir}

\lstinputlisting{../Readme_Recibir}

\subsection{Readme\_Reservar}

\lstinputlisting{../Readme_Reservar}

\subsection{Readme\_Grabar\_L}

\lstinputlisting{../Readme_Grabar_L}

\subsection{Readme\_Imprimir}

\lstinputlisting{../Readme_Imprimir}

\subsection{Readme\_Matar\_D}

\lstinputlisting{../Readme_Matar_D}

\newpage 


\section{Esquema de Ejecuci�n}

\section{Listado de nuevas funciones y/o comandos auxiliares}

\section{Listado de Nuevos Archivos}

\section{Listado de Datos}

\section{Prueba del Camino Feliz}
\end{document}
